\documentclass[11pt,a4paper]{scrartcl}
\usepackage[utf8]{inputenc}
\usepackage[english]{babel}
\usepackage{amsthm, amsmath, amscd}
\usepackage{enumerate}
\usepackage[hidelinks]{hyperref}

\newcommand{\of}[1]{\left(#1\right)}
\newcommand{\idm}[1]{\mathbf{1}_{#1}}
\newcommand{\cdef}[1]{\emph{#1}}
\newcommand{\code}[1]{\texttt{#1}}

\theoremstyle{plain}
\newtheorem{thm}{Theorem}
\theoremstyle{definition}
\newtheorem{defn}{Definition}
\newtheorem{exmp}{Example}
\theoremstyle{remark}
\newtheorem*{remark}{Remark}

\title{Category Theory Defintions for the \\Working Programmer }
\subtitle{Semi-formal category theory and its application to \\programming language type systems}
\author{Gudmundur F. Adalsteinsson}

\begin{document}
\maketitle
\begin{center}
	v0.1 \href{https://github.com/gummif/categories}{git source}
\end{center}
\abstract{The target audience for this document are programmers comfortable with abstact and high-level mathematics. A previous exposure to functional programming is a big plus. The text is very dry, but illustrative concrete examples are provided with code.}
\tableofcontents
\clearpage

\section{Introduction}

Consider the C++ programming language. It has types such as \texttt{int} and \texttt{double} and functions with signatures such as \texttt{int(int)} and \texttt{double(int)}. But programming languages are generally not pure and strict, and functions can take multiple arguments, have side effects, depend on memory addresses, use pointers, take and return references, be member functions, etc. So the C++ type system does not form a proper category (as defined below), although a very restricted portion might with limited functionality. Even type systems of pure functional programming languages like Haskell are not proper categories. 

Category theory still is a good mathematical model for computation. It provides powerful high-level abstractions and helps us reason about code.

\section{Categories}

\begin{defn}
A \cdef{category} $\mathcal{C}$ consists of
\begin{enumerate}
	\item a class of \cdef{objects} $\text{ob}(\mathcal{C})$
	\item a class of \cdef{arrows} or \cdef{morphisms} $\text{hom}(\mathcal{C})$. Every morphism has a source object and a target object. The class of morphisms between two objects $X$ and $Y$ are denoted $\text{hom}(X,Y)$.
	\item a binary operation $\circ$ for \cdef{morphism composition}
\end{enumerate}
with the following axioms
\begin{enumerate}[(i)]
	\item morphism composition is associative, $f \circ (g \circ h) = (f \circ g) \circ h$
	\item there exists an identity morphism $\idm{X}\colon X \to X$ (or $\text{id}_X$) such that $\idm{X} \circ f = f$ and $g \circ \idm{X} = g$ for all $f\colon A \to X$ and $g\colon X \to B$.
\end{enumerate}
\end{defn}

\begin{remark}
	The definition has no restriction on what objects and morphisms are. The names \emph{arrow} and \emph{morphism} are interchangable. We will use \emph{morphism} unless \emph{arrow} is more suitable.
\end{remark}
\begin{remark}
	If a category has a morphism from $X$ to $Y$ and a morphism from $Y$ to $Z$ then it must also contain the composed morphism from $X$ to $Z$.
\end{remark}

From the definition it can be shown that the identity morphism $\idm{X}$ for object $X$ is unique. For every category $\mathcal C$ we can construct the \cdef{dual category} $\mathcal{C}^\text{op}$ by reversing the arrows.

\begin{exmp}
	A common type of category is one where the objects are sets (or the types of a programming language) and the morphisms are functions. Another example is values and relations between values (e.g.\ the integers and the $\leq$ operator). Later we will see definitions where the objects are themselves categories and the morphisms are mappings between categories.
\end{exmp}

\begin{defn}
A morphism $f\colon X \to Y$ in $\mathcal C$ is an
\begin{itemize}
	\item \cdef{isomorphism}, if it has an inverse $g$ such that $g\circ f = \idm{X}$ and $f\circ g = \idm{Y}$
	\item \cdef{endomorphism}, if $X = Y$. The class $\text{hom}(X, X)$ is denoted $\text{end}(X)$.
	\item \cdef{automorphism}, if it's both an isomorphism and an endomorphism. The class of automorphisms of $X$ is denoted $\text{aut}(X)$.
\end{itemize}
\end{defn}

\section{Transformations}

Now that we have defined a category, we will take it up a notch and define a mapping between categories.

\begin{defn}
A \cdef{functor} $F$ is a mapping between categories $\mathcal{C}$ and $\mathcal{D}$, denoted $F\colon \mathcal{C} \to \mathcal{D}$, that 
\begin{enumerate}
	\item associates each object $X$ in $\mathcal{C}$ an object $F(X)$ in $\mathcal{D}$
	\item associates each morphism $f\colon X \to Y$ in $\mathcal{C}$ a morphism $F(f)\colon F(X) \to F(Y)$ in $\mathcal{D}$
\end{enumerate}
with the following axioms
\begin{enumerate}[(i)]
	\item preservation of composition, $F(g \circ f) = F(g) \circ F(f)$
	\item preservation of the identity morphism, $F(\idm{X}) = \idm{F(X)}$.
\end{enumerate}
\end{defn}

\begin{displaymath}
\begin{CD}
	\Big[ X @>{f}>> Y \Big] @>{F}>> \Big[ F\of X @>{F\of f}>> F\of Y \Big]
\end{CD}
\end{displaymath}

\begin{remark}
The nature of a functor $F$ is two-fold, both a mapping between objects and a mapping between morphisms.
\end{remark}

There exists an \cdef{identity functor} in $\mathcal{C}$ denoted $\idm{\mathcal{C}}$ mapping every object and morphism to itself. A functor that has the same category as domain and codomain is called an \cdef{endofunctor}. For a functor $F\colon \mathcal{C} \to \mathcal{D}$, if $\mathcal{D}$ is a subcategory of $\mathcal{C}$ we can safely say that $F\colon \mathcal{C} \to \mathcal{C}$ is an endofunctor. A composition of two functors $F\colon \mathcal{C} \to \mathcal{D}$ and $G\colon \mathcal{D} \to \mathcal{E}$ is simply denoted $GF\colon \mathcal{C} \to \mathcal{E}$. Since functors can be composed and there exists a unique identity functor for every category, given a category where the objects are themselves categories, the morphisms between the objects are functors.

\begin{exmp}
	Let's consider a parametric vector type \code{vector} (in pseudo-code). The type constructor forms the object mapping part of a functor, e.g.\ \code{vector} maps \code{string} to \code{vector<string>}. Now we simply need a parametric function of the form 
	\[\text{\code{vector<X>$\to$vector<Y> fmap(f:X$\to$Y)} }\]
	to define a functor. There is one obvious implementation of \code{fmap} that simply applies \code{f} to every element of the vector. Indeed, it can be shown that there is only one valid implementation of \code{fmap} satisfying the axioms. For languages that are not curried, the signature typically has an uncurried form 
	\[\text{\code{vector<Y> fmap(v:vector<X>, f:X$\to$Y)} }\]
	which is (isomorphically) equivalent.
\end{exmp}

If we now consider the functors from $\mathcal{C}$ to $\mathcal{D}$ as the objects of a category, we call the morpisms between those objects natural transformations.

TODO Natural transformation:
\begin{displaymath}
\begin{CD}
	F\of X @>{F\of f}>> F\of Y \\
	@V{\eta_X}VV @VV{\eta_Y}V \\
	G\of X @>{G\of f}>> G\of Y
\end{CD}
\end{displaymath}

\section{The M-word}

\end{document}
